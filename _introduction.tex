\chapter{Introduction}
With the rapid advancement of computer technology and robotics, we increasingly encounter entities in our daily lives that look extraordinarily human-like but are not real people. This can be observed mainly in the service sector and industry with the substantial increase in the use of robots, but also in the media industry, which creates exceptionally human-like entities with the help of advanced computer graphics for films, video games, and other media types. Many of these applications strive to achieve a design as similar as possible to that of a human being to increase the acceptance of these entities. However, here comes the uncanny valley into play. The uncanny valley effect proposed by Masahiro Mori \cite{original_masahiro} states that as the appearance of an entity is approaching, but failing to attain, a lifelike appearance, the empathy of a person towards the entity is abruptly shifting from affinity to revulsion. The implications of the uncanny valley thus have a profound impact on the evolution of human-like entities by attempting to explain the interplay between the affection and the appearance of human-like entities.\\
A current application of human-like artificial characters is robot recruiting. Robot recruiting describes the automation of the recruiting process in which the applicants are assessed and selected by algorithms and artificial intelligence. With the increasing capability of these algorithms and the resulting increase in interaction between algorithms and humans, a visual design can be chosen to represent these algorithms.
Although various different-looking robot recruiters are already being developed, there is still, up to my knowledge, no scientific research on the influence of the uncanny valley on visualised robot recruiters.\newpage
Therefore, this thesis is intended to give an overview of the uncanny valley effect and answer the specific research question:

\begin{quote}\emph{RQ 1: How do people regard the appearance of robot recruiters with a varying degree of human-likeness?}\end{quote} 

\begin{quote}\emph{RQ 2: Do very but no perfectly human-like robot recruiters fall into the uncanny valley?}\end{quote} 

\begin{quote}\emph{RQ 3: What level of human similarity should robot recruiters aim for in order to gain the most acceptance?}\end{quote} 

%\begin{quote}\emph{RQ 1: Do robot recruiters fall into the uncanny valley and how does this affect the intended human-likeness?}\end{quote} 

The first part of this thesis is a comprehensive literature review on the uncanny valley. With the help of existing literature, the effect itself will be explained, but also its origin and multiple different hypotheses on the uncanny valley effect will be discussed. The first part is concluded with an analysis of existing studies on the uncanny valley to provide an overview of the current state of research and possible recommendations for future research. (Chapter: \ref{chap:2},  Chapter: \ref{chap:3},  Chapter: \ref{chap:4},  Chapter: \ref{chap:5})\\
The second part of this thesis focuses on the research question. Using an online questionnaire, participants had to rate pictures of possible robot recruiters with varying human-likeness on four 7-point scales between polar adjectives: mechanical/human-like, artificial/lifelike, strange/familiar and not eerie/very eerie. In addition, the participants were asked how much they liked the picture of the robot recruiter. (Chapter: \ref{chap:6})\newline
The results of this survey will be used to assess the impact of the robot recruiter's appearance on the participants and determine whether human-like robot recruiters fall into the uncanny valley. From this, a suggestion may be made as to which degree of human similarity should be pursued in order to attain the highest level of robot recruiter acceptability.\\
The application process is essential for companies that want to filter out the best possible candidates and applicants who want to show their full potential. Robotic recruiting can assist businesses in making better and faster hiring decisions while removing the unconscious biases that occur in traditional human screening \cite{robot_recruiting_scholar}. As a result, both companies and applicants benefit from robot recruiting. However, candidates could be negatively influenced by a robot recruiter whose design falls into the uncanny valley, harming both the company and the applicant. Therefore, this thesis is relevant by providing a possible recommendation for the desired degree of human-likeness of visualised robot recruiters.