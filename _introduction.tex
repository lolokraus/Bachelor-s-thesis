\chapter{Introduction}
With the rapid advancement of computer technology and robotics, we increasingly encounter entities in our daily lives that look extraordinarily human-like but are not real people. This can be observed mainly in the service sector and industry with the substantial increase in the use of robots, but also in the media industry, which creates exceptionally human-like entities with the help of advanced computer graphics for films, video games, and other media types. All these applications strive to achieve a design as similar as possible to that of a human being to increase the acceptance of these entities. However, here comes the uncanny valley into play. The uncanny valley effect proposed by Masahiro Mori \cite{original_masahiro} states that as the appearance of an entity is approaching, but failing to attain, a lifelike appearance, the empathy of a person towards the entity is abruptly shifting from affinity to revulsion. The implications of the uncanny valley thus have a profound impact on the evolution of human-like entities by attempting to explain the interplay between the affection and the appearance of human-like figures.\\
A current application of human-like artificial characters is robot recruiting. Robot recruiting describes the automation of the recruiting process in which the applicants are assessed and selected by algorithms and artificial intelligence. The computer systems in the background are often depicted to applicants with human-like characters to give them a visual reference. Although various different looking robot recruiters are already being developed, there is still no scientific research on the influence of the uncanny valley on the acceptance of the look of these recruiters. Therefore, this thesis is intended to give an overview of the uncanny valley effect and answer the specific research question:

\begin{quote}\emph{How does the uncanny valley impact the acceptance and design recommendations of human-like robot recruiters?}\end{quote}

The first part of this paper is a comprehensive literature review on the uncanny valley. With the help of existing literature, the effect itself will be explained, but also its origin and multiple different hypotheses on the uncanny valley effect will be discussed. The first part is concluded with an analysis of existing studies on the uncanny valley to provide an overview of the current state of research and possible recommendations for future research.\\
The second part of this thesis will focus on the research question by extending existing research with a study on the influences of the uncanny valley on robot recruiting. The purpose of the study is to explore the effects of the uncanny valley concerning the changes in the behaviour of the applicants during the recruitment process toward human-like robot recruiters. The study consists of an online survey in which participants had to rate different pictures of possible robot recruiters of varying human likeness. The rating was done with four 7-point scales between polar adjectives: mechanical/humanlike, artificial/lifelike, strange/familiar and not eerie/very eerie. In addition, a question asking the participant how much they like the avatar was given.\\ 
Because of the lack of research in this direction and the rapid introduction of robot recruiting in many companies, the aim is to be able to make a statement about the acceptance of human-like robot recruiters and to make design suggestions for their appearance in the future. The application process is essential for companies that want to filter out the best possible candidates and applicants who want to show their full potential. Candidates could be negatively influenced by a robot recruiter whose design falls into the uncanny valley, harming both the company and the applicant. Therefore, it is crucial for companies to use robot recruiters wisely and to choose a meaningful design that makes the applicant feel comfortable. 