\chapter{Introduction}
With the rapid advancement of computer technology and robotics, we are increasingly encountering entities in our everyday lives that look extraordinarily humanlike but are not real people. This can be observed especially in the service sector and industry with the strong increase in the use of robots, but also in the media industry which creates exceptionally humanlike entities with the help of advanced computer graphics for films, video games and other media types. All these applications strive to achieve a design that is as similar as possible to that of a human being to increase the acceptance towards these entities. But here comes the uncanny valley effect into play. The uncanny valley effect proposed by Masahiro Mori \cite{original_masahiro} states that as the appearance of an entity is approaching, but failing to attain, a lifelike appearance the empathy of a person towards the entity would abruptly shift from affinity to revulsion. The implications of the uncanny valley thus have a profound impact on the evolution of humanlike entities by attempting to explain the interplay between the affection and the appearance of the entities.\\
The first part of this paper will introduce the phenomenon of the uncanny valley. With the help of existing literature, the effect itself will be explained, but also its origin and multiple different hypotheses on the uncanny valley effect will be discussed. The first part of the paper is concluded by a quantitative analysis of existing studies on the uncanny valley to give an overview of the current state of research and possible recommendations for future research.\\
The second part of this thesis extends the existing research with a study on the influences of the Uncanny Valley on robot recruiting. Robot recruiting describes the automation of the recruiting process in which the applicants are assessed and selected by algorithms and artificial intelligence. The computer systems in the background are often presented to the applicants through a visual end. The aim of the study to be conducted is to explore the applicants’ behaviour changes in the recruitment process  towards humalike visualised robot recruiting entities. 