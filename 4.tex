\chapter{A Review of Existing Research}
Nowadays, the uncanny valley has a broad influence on many disciplines with different research approaches. The previous chapters also show that the uncanny valley does not have a precise definition. A consequence of these factors is that studies tailor the stimuli, terms, assessment methods, and study participants to their exact research case, leading to inconsistent results in the conducted studies.
This chapter aims to provide an overview of these differences and describes the different stimuli, terms, evaluation methods and study participants of the studies mentioned in this paper, in addition to other relevant studies, following the structure of the quantitative analysis conducted by Jie Zhang et al. \cite{quant_review}.

 \section{Deficits of existing research}
 From the comparison of the studies described in this paper as well as other selected studies and the additional input of the literature review conducted by Jie Zhang et al. \cite{quant_review}, inconsistent conditions in existing studies could be found, which can lead to uncertain results. In the following, these conditions are described, and possible recommendations for their improvement are suggested. 

\subsection{Deviating and Inaccurate Terms}
Finding correct terms to describe feelings of social closeness, familiarity, attachment and affinity, or the opposite, feelings of fear, repulsion, aversion and eeriness towards robots and human-like entities is very difficult. Different people can interpret these terms, with which they are to evaluate entities, very individually. This can lead to a distortion of the real feelings towards the entities and thus also change the whole interpretation of the uncanny valley.\\
In the first formulation of the uncanny valley hypothesis in 1970 by Masahiro Mori \cite{original_masahiro_not_translated}, he used the two terms "shinwakan" and "bukimi" to describe the feelings towards different human replicas. The word "bukimi" could be clearly translated to eeriness, however, the word “shinwakan” was proven to be complex to translate \cite{quant_review}. At first, the word “shinwakan” was translated into familiarity by Jasia Reichardt \cite{first_translation}, but later, Masahiro Mori revised the translation into affinity. Furthermore, it has also been argued by \cite{uncanny_cliff} that "likeability" would be an even more appropriate translation. Based on the dictionary definitions of these terms, they all describe slightly different aspects of perceptual familiarity and emotional valence.\\ Given these terms' complex definitions and ambiguous character, a possible solution would be self-reporting questionnaires for measuring affinity. However, this option would not be feasible for every study due to the lack of uniformity and the resulting margin for interpretation of the given answers. So an essential step for future research would be to develop a common metric for the affinity dimension. The first step in this direction would be to agree on a translation of the term "shinwakan". Furthermore, studies also need to explore how participants perceive the expressions used and draw conclusions about how they value the different entities based on the expressions. 
%Tabelle

%%%%%
\subsection{A Wide Range of Stimuli}
Like describing feelings, it is difficult to give an exact definition of a figure's degree of human likeness. As a result, the selected stimuli can vary significantly between studies, and unnecessary additional variables are often introduced. 
%Table 1 compares the different stimuli from several studies.
It can be seen that participants have to evaluate a whole range of different stimuli, including images, videos and interactions and these often have different descriptions and are often staged differently. Moreover, often only a part of the entities is depicted, such as face, head or body.
Creating standardised stimuli for studies turns out to be very difficult. Researchers should not, select their entities arbitrarily or subjectively, because no accurate conclusions about the uncanny valley could be drawn. 
One way to solve the selection problem is to morph entities. This can create a meaningful comparative series from which good conclusions of human likeness and uncanny valley can be drawn. However, this approach cannot be used for every study. As Masahiro Mori has already assumed, the uncanny valley is not limited to appearance alone, but also to movement and other external influences. In order to explore these as well, studies need to use more complex stimuli. However, this leads back to the original problem.  In order to facilitate the selection, a kind of guideline for evaluating the human likeness of different entities would be needed in order to be able to objectively select the stimuli for studies. 
%abot database bewertung erwähnen
%stimme
%Tabelle
%%%%%%

\subsection{Participants from Different Cultures and Age Groups}
When selecting participants for a study, age-related and cultural differences, as well as personal differences, must be taken into account.\\
Yun-Chen et al. \cite{age_differences} conducted a thorough study examining the uncanny valley effect throughout different age groups and concluded that the uncanny valley is perceived very differently by participants of various age groups. Younger adults
preferred non-human-like robots, while middle-aged adults showed a preference for human-like robots. Surprisingly, the study even concluded that the uncanny valley was not observed in older adults. This would mean that the uncanny valley effect weakens with advancing age and possibly disappears completely. Considering the age-related differences found in this study, future research should take age as an essential factor that profoundly affects the uncanny valley. Due to the lack of comparative studies with older people or comparisons in different age groups, research in this direction is still very limited. Therefore research in this direction is urgently needed, especially concerning social robots for older adults.\\
Furthermore, most studies do not look into the participants' cultural differences and personal experiences. Other cultures have different ways of dealing with robots. In western culture, robots are portrayed as frightening machines \cite{japan_robot_friendly}. The opposite is true in Japan \cite{japan_robot_friendly}. Here, robots are much more accepted and widespread in society due to government promotion and a generally friendlier narrative \cite{japan_robot_friendly}. The effect of cultural, social and external influences on the uncanny valley has not yet been widely researched, and it is not unthinkable that they have a significant impact on the strength of the manifestation of the uncanny valley effect. The same applies to the personal exposure and experience to the uncanny valley. It could be assumed, for example, that a person who interacts with social robots in their everyday life does not feel the uncanny valley as strongly as a person who has never had contact with a robot.\\
Therefore, studies should pay attention to their participants' cultural, social, and personal environments. More research is needed in this direction to define the effects of these influences on the uncanny valley.




%Die Assessment Methoden?
%Aus den Vergleichen soll dann ein bestimmter Schluss gezogen werden (etwa, dass die Forschung sehr uneinheitlich ist und wahrscheinlich Bedarf an gewissen genormten Studien besteht)

