\chapter{A Review of Existing Research}
Nowadays, the uncanny valley has a broad influence on many disciplines with different research approaches. The previous chapters moreover show that the uncanny valley does not have an explicit definition. A consequence of these factors is that studies tailor the terms, stimuli, assessment methods, and study participants to their exact research case, leading to conflicting results, which further reinforce the indeterminacy of the uncanny valley.
This chapter aims to provide an overview of the inconsistencies between the studies mentioned in this paper and other relevant studies, following the structure of the literature review by Jie Zhang et al. \cite{quant_review}.

 \section{Deficits of existing research}
 By comparing the studies described in this paper as well as other relevant studies and with the additional input of the literature review conducted by Jie Zhang et al. \cite{quant_review}, 
 it can be recognized that the inconsistent conditions in existing studies can lead to deficits in their research. These deficits cause conflicting results and therefore complicate a uniform definition of the uncanny valley. The following describes the significant areas in which the majority of differing conditions occur. The aim is to identify and give an overview of these areas and also make possible recommendations that could unify or improve future research. 

\subsection{Deviating and Inaccurate Terms}
Finding correct terms to describe feelings of social closeness, familiarity, attachment and affinity, or the opposite, feelings of fear, repulsion, aversion and eeriness towards robots and human-like entities is very difficult. Different people can interpret these terms, with which they are to evaluate entities, very individually. This can lead to a distortion of the real feelings toward the entities and thus also change the whole interpretation of the uncanny valley.
\begin{table}
\centering
\setlength{\extrarowheight}{0pt}
\addtolength{\extrarowheight}{\aboverulesep}
\addtolength{\extrarowheight}{\belowrulesep}
\setlength{\aboverulesep}{0pt}
\setlength{\belowrulesep}{0pt}
\resizebox{\linewidth}{!}{%
\begin{tabular}{|c|l|l|} 
\toprule
\rowcolor[rgb]{0.753,0.753,0.753} \multicolumn{1}{|l|}{\textbf{Mori's Original Terms}} & \textbf{Used Terms} & \textbf{Author and Year} \\ 
\hline
\multirow{8}{*}{\begin{tabular}[c]{@{}c@{}}\textbf{Positive Feelings}\\\textbf{"shinwakan"}\\\textbf{recommended translation "affinity"}\end{tabular}} & familiarity & \begin{tabular}[c]{@{}l@{}}\cite{uncanny_ambiguous} \\\cite{uncanny_wall}\end{tabular} \\ 
\cline{2-3}
 & likeability & \begin{tabular}[c]{@{}l@{}}\cite{childrens_responding}\\\cite{age_differences}\\\cite{uncanny_cliff}\end{tabular} \\ 
\cline{2-3}
 & affinity & \begin{tabular}[c]{@{}l@{}}\cite{original_masahiro}\\\cite{review_4}\end{tabular} \\ 
\cline{2-3}
 & nice & \cite{uncanny_cliff} \\ 
\cline{2-3}
 & friendly & \cite{uncanny_cliff} \\ 
\cline{2-3}
 & kind & \cite{uncanny_cliff} \\ 
\cline{2-3}
 & pleasant & \cite{uncanny_cliff} \\ 
\cline{2-3}
 & attractive & \begin{tabular}[c]{@{}l@{}}\cite{review_1}\\\cite{review_3}\end{tabular} \\
\cline{2-3}
 & reassuring & \cite{review_3} \\ 
\hline
\multirow{9}{*}{\begin{tabular}[c]{@{}c@{}}\textbf{Negative Feelings}\\\textbf{"bukimi"}\\\textbf{recommended translation "eeriness"}\end{tabular}}
 & eerie & \begin{tabular}[c]{@{}l@{}}\cite{uncanny_ambiguous} \\\cite{uncanny_wall} \\\cite{prior_exposure_robots} \\\cite{review_1} \\\cite{review_2} \\\cite{review_3}\\\end{tabular} \\ 
\cline{2-3}
& strange & \begin{tabular}[c]{@{}l@{}}\cite{uncanny_ambiguous} \\\cite{uncanny_wall}\end{tabular} \\ 
\cline{2-3}
 & awful & \cite{uncanny_cliff} \\ 
\cline{2-3}
 & unfriendly & \cite{uncanny_cliff} \\ 
\cline{2-3}
 & unkind & \cite{uncanny_cliff} \\ 
\cline{2-3}
 & unpleasant & \cite{uncanny_cliff} \\ 
\cline{2-3}
 & dislike & \cite{childrens_responding} \\ 
\cline{2-3}
 & disgust & \cite{age_differences} \\ 
\cline{2-3}
 & unattractive & \cite{review_3} \\
\bottomrule
\end{tabular}
}
\caption{Various terms used in different studies.}
\label{tab:terms}
\end{table}

Table \ref{tab:terms} shows that studies use a wide variety of terms to express positive or negative feelings toward the adopted entities. Masahiro Mori used in his first formulation of the uncanny valley hypothesis in 1970  \cite{original_masahiro_not_translated} the two terms "shinwakan" and "bukimi" to describe the positive and negative feelings towards different human replicas. The word "bukimi" could be clearly translated to eeriness, however, the word “shinwakan” was proven to be complex to translate \cite{quant_review}. At first, the word “shinwakan” was translated into familiarity by Jasia Reichardt \cite{first_translation}, but later, Masahiro Mori revised the translation into affinity. Furthermore, it has also been argued by \cite{uncanny_cliff} that "likeability" would be an even more appropriate translation. Based on the dictionary definitions of these terms, they all describe slightly different aspects of perceptual familiarity and emotional valence.\\
This translation barrier can be seen well in the summary of the expressions in table \ref{tab:terms}. The negative word "eerie" was used in most of the studies. Sometimes in addition with other descriptive adjectives. In contrast, the recommended translation "affinity" for positive feelings, has been used by only a few studies and does not seem to be accepted yet. The literature analysis by \cite{quant_review} additionally to the table \ref{tab:terms} shows evidently that the first translation "familiarity" is still more favoured and together with simple liking questions, is one of the most commonly used ways of testing positive feelings toward entities.\\ %Die Assessment Methoden? TODO
A possible alternative to using predefined expressions for the assessment would be self-reporting questionnaires for measuring affinity. However, this option would not be feasible for every study due to the lack of uniformity and the resulting margin for interpretation of the given answers.\\
Therefore an essential step for future research would be to develop a common metric for the affinity dimension. The first step in this direction would be to agree on a translation of the term "shinwakan". Furthermore, studies also need to explore how participants perceive the expressions used and draw conclusions about how they perceive the different entities based on their valuation of the given expressions. 

\newpage
%%%%%
\subsection{A Wide Range of Stimuli}
Like describing feelings, it is difficult to give an exact definition of a figure's degree of human likeness. As a result, the selected stimuli can vary significantly between studies, and unnecessary additional variables are often introduced. 
%Table 1 compares the different stimuli from several studies.
It can be seen that participants have to evaluate a whole range of different stimuli, including images, videos and interactions and these often have different descriptions and are often staged differently. Moreover, often only a part of the entities is depicted, such as face, head or body.
Creating standardised stimuli for studies turns out to be very difficult. Researchers should not, select their entities arbitrarily or subjectively, because no accurate conclusions about the uncanny valley could be drawn. 
One way to solve the selection problem is to morph entities. This can create a meaningful comparative series from which good conclusions of human likeness and uncanny valley can be drawn. However, this approach cannot be used for every study. As Masahiro Mori has already assumed, the uncanny valley is not limited to appearance alone, but also to movement and other external influences. In order to explore these as well, studies need to use more complex stimuli. However, this leads back to the original problem.  In order to facilitate the selection, a kind of guideline for evaluating the human likeness of different entities would be needed in order to be able to objectively select the stimuli for studies. 
%abot database bewertung erwähnen
%stimme
%Tabelle
%%%%%%
\clearpage
\newpage

\begin{sidewaystable}
\centering
\resizebox{\linewidth}{!}{%
\begin{tabular}{|l|l|l|l|l|l|} 
\hline
\rowcolor[rgb]{0.753,0.753,0.753}  & Assesement: & Entities: & Picture/Video: & Independent/Morphed:~ ~ & Body Parts: \\ 
\hline
{\cellcolor[rgb]{0.753,0.753,0.753}}\cite{uncanny_ambiguous} & \begin{tabular}[c]{@{}l@{}}individual computerbased\\questionnaire\end{tabular} & \begin{tabular}[c]{@{}l@{}}14 robots with \\variing human likeness\end{tabular} & Video & Independent & \begin{tabular}[c]{@{}l@{}}Whole Body,\\Head/Face,\\Extremity\end{tabular} \\ 
\hline
{\cellcolor[rgb]{0.753,0.753,0.753}}\cite{uncanny_cliff} & \begin{tabular}[c]{@{}l@{}}individual~computerbased\\questionnaire\end{tabular} & \begin{tabular}[c]{@{}l@{}}18 robot like entities that either\\exist or that are extremely \\similar to existing entities\end{tabular} & Picture & Independent & \begin{tabular}[c]{@{}l@{}}Whole Body,\\Head/Face\end{tabular} \\ 
\hline
{\cellcolor[rgb]{0.753,0.753,0.753}}\cite{uncanny_wall} & web-based questionnaire & \begin{tabular}[c]{@{}l@{}}14 videos of video game\\characters\end{tabular} & Video & Independent & \begin{tabular}[c]{@{}l@{}}Whole Body,\\Head/Face\end{tabular} \\ 
\hline
{\cellcolor[rgb]{0.753,0.753,0.753}}\cite{age_differences} & web-based questionnaire & 83 pictures of robots & Picture & Independent & Head/Face \\ 
\hline
{\cellcolor[rgb]{0.753,0.753,0.753}}\cite{review_3} & questionnaire with scales & \begin{tabular}[c]{@{}l@{}}MetaHumans created using the \\Unreal Engine online \\application MetaHuman Creator\end{tabular} & Video & Independent & Head/Face \\ 
\hline
{\cellcolor[rgb]{0.753,0.753,0.753}}\cite{review_1} & questionnaire with scales & \begin{tabular}[c]{@{}l@{}}12 video clips of \\human-like characters\end{tabular} & Video & Independent & \begin{tabular}[c]{@{}l@{}}Whole Body,\\Head/Face\end{tabular} \\ 
\hline
{\cellcolor[rgb]{0.753,0.753,0.753}}\cite{prior_exposure_robots} & \begin{tabular}[c]{@{}l@{}}viewing and rating\\assesment\end{tabular} & \begin{tabular}[c]{@{}l@{}}50 photographs~of a distinct \\robot or human agent\end{tabular} & Picture & Independent & \begin{tabular}[c]{@{}l@{}}Head/Face\\Extremity\end{tabular} \\ 
\hline
{\cellcolor[rgb]{0.753,0.753,0.753}}\cite{review_2} & \begin{tabular}[c]{@{}l@{}}viewing and rating \\assesment\end{tabular} & \begin{tabular}[c]{@{}l@{}}60 distinct agents which spanned \\two ontological categories \\(robot, person)~ ~\end{tabular} & Picture & Independent & \begin{tabular}[c]{@{}l@{}}Whole Body,\\Head/Face\end{tabular} \\ 
\hline
{\cellcolor[rgb]{0.753,0.753,0.753}}\cite{uncanny_infants} & recorded visual fixations & \begin{tabular}[c]{@{}l@{}}human face,\\realistic avatar face,\\uncanny avatar face \\uttering the syllable /ba/ silently~ ~\end{tabular} & Video & Independent & Head/Face \\ 
\hline
{\cellcolor[rgb]{0.753,0.753,0.753}}\cite{uncanny_infant_discrimination} & recorded visual fixations & \begin{tabular}[c]{@{}l@{}}a human, an android, \\a mechanical robot\\performing agrasping \\action with their right hand\end{tabular} & Video & Independent & Whole Body \\ 
\hline
{\cellcolor[rgb]{0.753,0.753,0.753}}\cite{childrens_responding} & \begin{tabular}[c]{@{}l@{}}viewing duration, \\termination frequency,\\liking question\end{tabular} & \begin{tabular}[c]{@{}l@{}}24 photographs \\of agents of varying \\human likeness\end{tabular} & Picture & Independent & \begin{tabular}[c]{@{}l@{}}Whole Body,\\Head/Face\end{tabular} \\
\hline
\end{tabular}
}
\caption{Various stimuli used in different studies.}
\label{tab:stimuli}
\end{sidewaystable}
\clearpage
\newpage
\subsection{Participants from Different Cultures and Age Groups}
When selecting participants for a study, age-related and cultural differences, as well as personal differences, must be taken into account.\\
Yun-Chen et al. \cite{age_differences} conducted a thorough study examining the uncanny valley effect throughout different age groups and concluded that the uncanny valley is perceived very differently by participants of various age groups. Younger adults
preferred non-human-like robots, while middle-aged adults showed a preference for human-like robots. Surprisingly, the study even concluded that the uncanny valley was not observed in older adults. This would mean that the uncanny valley effect weakens with advancing age and possibly disappears completely. Considering the age-related differences found in this study, future research should take age as an essential factor that profoundly affects the uncanny valley. Due to the lack of comparative studies with older people or comparisons in different age groups, research in this direction is still very limited. Therefore research in this direction is urgently needed, especially concerning social robots for older adults.\\
Furthermore, most studies do not look into the participants' cultural differences and personal experiences. Other cultures have different ways of dealing with robots. In western culture, robots are portrayed as frightening machines \cite{japan_robot_friendly}. The opposite is true in Japan \cite{japan_robot_friendly}. Here, robots are much more accepted and widespread in society due to government promotion and a generally friendlier narrative \cite{japan_robot_friendly}. The effect of cultural, social and external influences on the uncanny valley has not yet been widely researched, and it is not unthinkable that they have a significant impact on the strength of the manifestation of the uncanny valley effect. The same applies to the personal exposure and experience to the uncanny valley. It could be assumed, for example, that a person who interacts with social robots in their everyday life does not feel the uncanny valley as strongly as a person who has never had contact with a robot.\\
Tables \ref{tab:participation_1} and \ref{tab:participation_2} list the number of participants, their gender, age, cultural background and occupation. Often studies do not explicitly address these factors and thus do not take them into account at all. Only studies that examine one of these aspects in particular, such as \cite{childrens_responding}, \cite{uncanny_infants}, \cite{uncanny_infant_discrimination} and \cite{age_differences} deliberately select them and investigate them in greater depth. However, it can be deduced from research of these studies that these factors can always influence the results of the studies and thus, in the worst case, falsify the results. 
Therefore, it can be argued that studies should always collect data on these factors and present the uncanny valley only in the context of these factors. Moreover more research is needed in this direction to better define the effects of these influences on the uncanny valley to get a more accurate picture of the uncanny valley.

\newpage
\begin{sidewaystable}
%\centering
\setlength{\extrarowheight}{0pt}
\addtolength{\extrarowheight}{\aboverulesep}
\addtolength{\extrarowheight}{\belowrulesep}
\setlength{\aboverulesep}{0pt}
\setlength{\belowrulesep}{0pt}
\resizebox{\linewidth}{!}{%
\begin{tabular}{|l|l|l|l|l|l|l|l|l|l|l|l|} 
\toprule
\rowcolor[rgb]{0.753,0.753,0.753}  & \cite{uncanny_ambiguous} & \cite{uncanny_cliff} & \cite{uncanny_wall} & \cite{uncanny_infants} & \cite{uncanny_infant_discrimination} & \cite{childrens_responding} \\ 
\hline
{\cellcolor[rgb]{0.753,0.753,0.753}}Participants: & 56 & 58 & 100 & 96 & 42 & 77 \\ 
\hline
{\cellcolor[rgb]{0.753,0.753,0.753}}Female: & 13 & 28 & 8 & 46 & 22 & 38 \\ 
\hline
{\cellcolor[rgb]{0.753,0.753,0.753}}Male: & 43 & 30 & 92 & 50 & 20 & 39 \\ 
\hline
{\cellcolor[rgb]{0.753,0.753,0.753}}Age: & 17-35 years & 18-41 years & not disclosed & 6-12 months & 6-14 months & not disclosed \\ 
\hline
{\cellcolor[rgb]{0.753,0.753,0.753}}Culture: & Indonesian & \begin{tabular}[c]{@{}l@{}}presumed~Japanese\\(not directly disclosed)\end{tabular} & \begin{tabular}[c]{@{}l@{}}presumed American\\(not directly disclosed)\end{tabular} & mostly Caucasian & not disclosed & \begin{tabular}[c]{@{}l@{}}presumed American\\(not directly disclosed)\end{tabular} \\ 
\hline
{\cellcolor[rgb]{0.753,0.753,0.753}}Occupation: & \begin{tabular}[c]{@{}l@{}}university students,\\young professionals,\\government workers~\end{tabular} & \begin{tabular}[c]{@{}l@{}}participants were\\associated with a\\university in the\\Kyoto district of\\Japan\end{tabular} & \begin{tabular}[c]{@{}l@{}}university students\\from the School of\\Games Computing\\and Creative\\Technologies at\\Bolton University~ ~\end{tabular} &  &  & \\
\bottomrule
\end{tabular}
}
\caption{Various participation groups.}
\label{tab:participation_1}
%\end{sidewaystable}

%\begin{sidewaystable}
%\centering
\setlength{\extrarowheight}{0pt}
\addtolength{\extrarowheight}{\aboverulesep}
\addtolength{\extrarowheight}{\belowrulesep}
\setlength{\aboverulesep}{0pt}
\setlength{\belowrulesep}{0pt}
\resizebox{\linewidth}{!}{%
\begin{tabular}{|l|l|l|l|l|l|l|l|l|l|l|l|} 
\toprule
\rowcolor[rgb]{0.753,0.753,0.753} & \cite{prior_exposure_robots} & \cite{age_differences} & \cite{review_1} & \cite{review_2} & \cite{review_3} \\ 
\hline
{\cellcolor[rgb]{0.753,0.753,0.753}}Participants: & 86 & 255 & 30 & 72 & 20 \\ 
\hline
{\cellcolor[rgb]{0.753,0.753,0.753}}Female: & not disclosed & 157 & 9 & 46 & 10 \\ 
\hline
{\cellcolor[rgb]{0.753,0.753,0.753}}Male: & not disclosed & 98 & 21 & 26 & 10 \\ 
\hline
{\cellcolor[rgb]{0.753,0.753,0.753}}Age: & not disclosed & 18-65+ years & median age: 26 years & 18-49 years & not disclosed \\ 
\hline
{\cellcolor[rgb]{0.753,0.753,0.753}}Culture: & not disclosed & mostly Taiwanese & not disclosed & \begin{tabular}[c]{@{}l@{}}presumed American\\(not directly disclosed)\end{tabular} & not disclosed \\ 
\hline
{\cellcolor[rgb]{0.753,0.753,0.753}}Occupation: & not disclosed & not disclosed & not disclosed & \begin{tabular}[c]{@{}l@{}}recruited from Tufts\\University and the\\surrounding community\end{tabular} & \begin{tabular}[c]{@{}l@{}}recruited from\\a combination\\of university\\mailing lists\\and prolific\\participant\\recruitment\end{tabular} \\
\bottomrule
\end{tabular}
}
\caption{Various participation groups.}
\label{tab:participation_2}
\end{sidewaystable}