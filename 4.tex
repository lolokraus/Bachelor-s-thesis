\chapter{A Review of Existing Research}
\label{chap:5}
Nowadays, the uncanny valley is explored within many different research fields with various research approaches. The previous chapters moreover showed that the uncanny valley does not have an explicit definition. A consequence of these factors is that studies tailor the terms, stimuli, assessment methods, and study participants to their exact research case, leading to conflicting results that further reinforce the uncanny valley's indeterminacy.
This chapter aims to provide an overview of the inconsistencies between the studies mentioned in this paper and other relevant studies, following the structure of the literature review by Jie Zhang et al. \cite{quant_review}.

 \section{Deficits of existing research}
 By comparing the studies described in this thesis so far as well as other relevant studies, which were selected from the last five years based on the frequency of citations, and with the additional input of the literature review conducted by Jie Zhang et al. \cite{quant_review}, it can be recognized that the inconsistent conditions in existing studies can lead to deficits in their research. These deficits cause conflicting results and therefore complicate a uniform definition of the uncanny valley. The following describes the significant areas in which the majority of differing conditions occur. The aim is to identify and give an overview of these areas and make possible recommendations that could unify or improve future research. 

\subsection{Deviating and Inaccurate Terms}
Finding correct terms to describe feelings of social closeness, familiarity, attachment and affinity, or the opposite, feelings of fear, repulsion, aversion and eeriness towards robots and human-like entities, is challenging. Different people can interpret these terms, with which they are to evaluate entities individually. This could distort the real feelings toward the entities and thus also change the whole interpretation of the uncanny valley.
\begin{table}
\centering
\setlength{\extrarowheight}{0pt}
\addtolength{\extrarowheight}{\aboverulesep}
\addtolength{\extrarowheight}{\belowrulesep}
\setlength{\aboverulesep}{0pt}
\setlength{\belowrulesep}{0pt}
\resizebox{\linewidth}{!}{%
\begin{tabular}{|c|l|l|} 
\toprule
\rowcolor[rgb]{0.753,0.753,0.753} \multicolumn{1}{|l|}{\textbf{Mori's Original Terms}} & \textbf{Used Terms} & \textbf{Author and Year} \\ 
\hline
\multirow{8}{*}{\begin{tabular}[c]{@{}c@{}}\textbf{Positive Feelings}\\\textbf{`shinwakan'}\\\textbf{recommended translation `affinity'}\end{tabular}} & familiarity & \begin{tabular}[c]{@{}l@{}}\cite{uncanny_ambiguous} \\\cite{uncanny_wall}\end{tabular} \\ 
\cline{2-3}
 & likeability & \begin{tabular}[c]{@{}l@{}}\cite{childrens_responding}\\\cite{age_differences}\\\cite{uncanny_cliff}\end{tabular} \\ 
\cline{2-3}
 & affinity & \begin{tabular}[c]{@{}l@{}}\cite{original_masahiro}\\\cite{review_4}\end{tabular} \\ 
\cline{2-3}
 & nice & \cite{uncanny_cliff} \\ 
\cline{2-3}
 & friendly & \cite{uncanny_cliff} \\ 
\cline{2-3}
 & kind & \cite{uncanny_cliff} \\ 
\cline{2-3}
 & pleasant & \cite{uncanny_cliff} \\ 
\cline{2-3}
 & attractive & \begin{tabular}[c]{@{}l@{}}\cite{review_1}\\\cite{review_3}\end{tabular} \\
\cline{2-3}
 & reassuring & \cite{review_3} \\ 
\hline
\multirow{9}{*}{\begin{tabular}[c]{@{}c@{}}\textbf{Negative Feelings}\\\textbf{`bukimi'}\\\textbf{recommended translation `eeriness'}\end{tabular}}
 & eerie & \begin{tabular}[c]{@{}l@{}}\cite{uncanny_ambiguous} \\\cite{uncanny_wall} \\\cite{prior_exposure_robots} \\\cite{review_1} \\\cite{review_2} \\\cite{review_3}\\\end{tabular} \\ 
\cline{2-3}
& strange & \begin{tabular}[c]{@{}l@{}}\cite{uncanny_ambiguous} \\\cite{uncanny_wall}\end{tabular} \\ 
\cline{2-3}
 & awful & \cite{uncanny_cliff} \\ 
\cline{2-3}
 & unfriendly & \cite{uncanny_cliff} \\ 
\cline{2-3}
 & unkind & \cite{uncanny_cliff} \\ 
\cline{2-3}
 & unpleasant & \cite{uncanny_cliff} \\ 
\cline{2-3}
 & dislike & \cite{childrens_responding} \\ 
\cline{2-3}
 & disgust & \cite{age_differences} \\ 
\cline{2-3}
 & unattractive & \cite{review_3} \\
\bottomrule
\end{tabular}
}
\caption{Various terms used in different studies.}
\label{tab:terms}
\end{table}

With the collected expressions table, \ref{tab:terms} shows that a wide variety of terms is used to express positive or negative feelings toward the adopted entities. Masahiro Mori used in his first formulation of the uncanny valley hypothesis in 1970  \cite{original_masahiro_not_translated} the two terms `shinwakan' and `bukimi' to describe the positive and negative feelings towards different human replicas. The word `bukimi' could be clearly translated to eeriness, however, the word “shinwakan” was proven to be complex to translate \cite{quant_review}. At first, the word “shinwakan” was translated into familiarity by Jasia Reichardt \cite{first_translation}, but later, Masahiro Mori revised the translation into affinity. Furthermore, it has also been argued by \cite{uncanny_cliff} that `likeability' would be an even more appropriate translation. Based on the dictionary definitions of these terms, they all describe slightly different aspects of perceptual familiarity and emotional valence.\\
This translation barrier can be noticed well in the summary of the expressions in table \ref{tab:terms}. The negative word `eerie' was used in most of the studies. Sometimes, in addition to other descriptive adjectives. In contrast, the recommended translation `affinity' for positive feelings has been used by only a few studies and does not seem to be accepted yet. The literature analysis by \cite{quant_review}, additionally to the used expressions in table \ref{tab:terms} shows evidently that the first translation, `familiarity' is still more favoured and, together with simple liking questions, it belongs to the most commonly used ways of testing positive feelings toward entities.\\
A possible alternative to using predefined expressions for the assessment would be self-reporting questionnaires for measuring affinity. However, this option is not feasible for every study due to the lack of uniformity and the resulting margin for interpretation of the given answers.\\
Therefore an essential step for future research is to develop a common metric for the affinity dimension. The first step in this direction is to agree on a translation of the term `shinwakan'. Furthermore, studies also need to explore how participants perceive the expressions used and draw conclusions about how they perceive the different entities based on their valuation of the given expressions. 
\newpage

\subsection{A Wide Range of Stimuli}
Different studies naturally use different stimuli for their research. As shown in table \ref{tab:stimuli}, which lists the details of the used stimuli and the study assessment methods, study participants are often presented with independent videos or pictures of robots or computer-generated characters with varying human-likeness \cite{quant_review}. This makes these selected entities a very imprecise and highly variable object of assessment. In addition, just like the uncanny feelings, human-likeness is a complex variable without a unified definition \cite{quant_review}.\\
When selecting entities, it can be seen from the summary of table \ref{tab:stimuli} that studies use both real-life robots and computer-generated characters. Furthermore, only a part of an entity is often shown, such as the face, whole head or extremities. Showing individual body parts could distort the uncanny valley effect for an entity with an entire body, as the remaining body parts could negatively or positively affect the uncanny valley \cite{quant_review}. Participants were shown either videos or pictures of the entities in the selected studies. However, other studies also introduced the entities to the participants in other ways, such as through descriptions, words, or even interactions \cite{quant_review}. As Masahiro Mori \cite{original_masahiro} already hypothesised, the movement of entities caused by videos of them or interactions with them strongly influences the uncanny valley, and other forms of presentation could also have a strong influence on the uncanny valley. Therefore, studies should explain precisely why they chose which form of presentation and what impact they observed or expected it to have on the uncanny valley.\\
To determine the uncanny valley, the human-likeness of an entity must be specified. This can pose a problem because, as already mentioned, human-likeness does not have a unified definition.
One possible solution would be the morphing of entities. This can create a meaningful comparative series from which sound conclusions of human-likeness in relation to the uncanny valley can be drawn \cite{quant_review}. However, this approach cannot be used for every study, and the morphing would have to increase in a controlled way exactly with the human resemblance \cite{quant_review}.
To define the human-likeness, another standardised way of assessing the entities is needed.
An attempt to define human-likeness more precisely has been made by the ABOT Database (https://www.abotdatabase.info/), which generates a human-likeness score with its `Human-Likeness Predictor' by asking several questions about the appearance of a robot.
Currently, this predictor gives only a rough score of human-likeness based on various questions about the appearance of the entity but adapting and expanding this predictor could be a possible approach to solving the problem. Ultimately researchers should analyse the selected entities more thoroughly and objectively analyse their human similarities in order to create more accurate conclusions for their research.

% Finally, differences can also be recognised in the evaluation options. Often participants have to evaluate the entities with predefined terms, but sometimes the liking or disliking by filming the eye movements is also used for evaluation. 
\clearpage
\newpage

\begin{sidewaystable}
\centering
\resizebox{\linewidth}{!}{%
\begin{tabular}{|l|l|l|l|l|l|} 
\hline
\rowcolor[rgb]{0.753,0.753,0.753}  & Assessment: & Entities: & Picture/Video: & Independent/Morphed:~ ~ & Body Parts: \\ 
\hline
{\cellcolor[rgb]{0.753,0.753,0.753}}\cite{uncanny_ambiguous} & \begin{tabular}[c]{@{}l@{}}individual compute-rbased\\questionnaire\end{tabular} & \begin{tabular}[c]{@{}l@{}}14 robots with \\variing human-likeness\end{tabular} & Video & Independent & \begin{tabular}[c]{@{}l@{}}Whole Body,\\Head/Face,\\Extremity\end{tabular} \\ 
\hline
{\cellcolor[rgb]{0.753,0.753,0.753}}\cite{uncanny_cliff} & \begin{tabular}[c]{@{}l@{}}individual~computer-based\\questionnaire\end{tabular} & \begin{tabular}[c]{@{}l@{}}18 robot like entities that either\\exist or that are extremely \\similar to existing entities\end{tabular} & Picture & Independent & \begin{tabular}[c]{@{}l@{}}Whole Body,\\Head/Face\end{tabular} \\ 
\hline
{\cellcolor[rgb]{0.753,0.753,0.753}}\cite{uncanny_wall} & web-based questionnaire & \begin{tabular}[c]{@{}l@{}}14 videos of video game\\characters\end{tabular} & Video & Independent & \begin{tabular}[c]{@{}l@{}}Whole Body,\\Head/Face\end{tabular} \\ 
\hline
{\cellcolor[rgb]{0.753,0.753,0.753}}\cite{age_differences} & web-based questionnaire & 83 pictures of robots & Picture & Independent & Head/Face \\ 
\hline
{\cellcolor[rgb]{0.753,0.753,0.753}}\cite{review_3} & questionnaire with scales & \begin{tabular}[c]{@{}l@{}}MetaHumans created using the \\Unreal Engine online \\application MetaHuman Creator\end{tabular} & Video & Independent & Head/Face \\ 
\hline
{\cellcolor[rgb]{0.753,0.753,0.753}}\cite{review_1} & questionnaire with scales & \begin{tabular}[c]{@{}l@{}}12 video clips of \\human-like characters\end{tabular} & Video & Independent & \begin{tabular}[c]{@{}l@{}}Whole Body,\\Head/Face\end{tabular} \\ 
\hline
{\cellcolor[rgb]{0.753,0.753,0.753}}\cite{prior_exposure_robots} & \begin{tabular}[c]{@{}l@{}}viewing and rating\\assesment\end{tabular} & \begin{tabular}[c]{@{}l@{}}50 photographs~of a distinct \\robot or human agent\end{tabular} & Picture & Independent & \begin{tabular}[c]{@{}l@{}}Head/Face\\Extremity\end{tabular} \\ 
\hline
{\cellcolor[rgb]{0.753,0.753,0.753}}\cite{review_2} & \begin{tabular}[c]{@{}l@{}}viewing and rating \\assesment\end{tabular} & \begin{tabular}[c]{@{}l@{}}60 distinct agents which spanned \\two ontological categories \\(robot, person)~ ~\end{tabular} & Picture & Independent & \begin{tabular}[c]{@{}l@{}}Whole Body,\\Head/Face\end{tabular} \\ 
\hline
{\cellcolor[rgb]{0.753,0.753,0.753}}\cite{uncanny_infants} & recorded visual fixations & \begin{tabular}[c]{@{}l@{}}human face,\\realistic avatar face,\\uncanny avatar face \\uttering the syllable /ba/ silently~ ~\end{tabular} & Video & Independent & Head/Face \\ 
\hline
{\cellcolor[rgb]{0.753,0.753,0.753}}\cite{uncanny_infant_discrimination} & recorded visual fixations & \begin{tabular}[c]{@{}l@{}}a human, an android, \\a mechanical robot\\performing a grasping \\action with their right hand\end{tabular} & Video & Independent & Whole Body \\ 
\hline
{\cellcolor[rgb]{0.753,0.753,0.753}}\cite{childrens_responding} & \begin{tabular}[c]{@{}l@{}}viewing duration, \\termination frequency,\\liking question\end{tabular} & \begin{tabular}[c]{@{}l@{}}24 photographs \\of agents of varying \\human-likeness\end{tabular} & Picture & Independent & \begin{tabular}[c]{@{}l@{}}Whole Body,\\Head/Face\end{tabular} \\
\hline
\end{tabular}
}
\caption{Various stimuli used in different studies.}
\label{tab:stimuli}
\end{sidewaystable}
\clearpage
\newpage
\subsection{Participants from Different Cultures and Age Groups}
When selecting participants for a study, age-related and cultural differences, as well as personal differences, must be taken into account.\\
Yun-Chen et al. \cite{age_differences} conducted a thorough study examining the uncanny valley effect throughout different age groups and concluded that the uncanny valley is perceived very differently by participants of various age groups. Younger adults
preferred non-human-like robots, while middle-aged adults showed a preference for human-like robots. Surprisingly, the study even concluded that the uncanny valley was not observed in older adults. This would mean that the uncanny valley effect weakens with advancing age and possibly disappears completely. Considering the age-related differences found in this study, future research should take age as an essential factor that profoundly affects the uncanny valley. Due to the lack of comparative studies with older people or comparisons in different age groups, research in this area is still very limited. Therefore research in this direction is urgently needed, especially concerning social robots for older adults.\\
Furthermore, most studies do not consider the participants' cultural differences and personal experiences. Other cultures have different ways of dealing with robots. In western culture, robots are portrayed as frightening machines \cite{japan_robot_friendly}. The opposite is true in Japan \cite{japan_robot_friendly}. Here, robots are much more accepted and widespread in society due to government promotion and a generally friendlier narrative \cite{japan_robot_friendly}. The effect of cultural, social and external influences on the uncanny valley has not yet been widely researched, and it is not unthinkable that they have a significant impact on the strength of the manifestation of the uncanny valley effect. The same applies to the personal exposure and experience of the uncanny valley. It could be assumed, for example, that a person who interacts with social robots in their everyday life does not feel the uncanny valley as strongly as a person who has never had contact with a robot.\\
Tables \ref{tab:participation_1} and \ref{tab:participation_2} lists the number of participants, their gender, age, cultural background and occupation. Often studies do not explicitly address these factors and thus do not take them into account. Only studies that examine one of these aspects in particular, such as \cite{childrens_responding}, \cite{uncanny_infants}, \cite{uncanny_infant_discrimination} and \linebreak \cite{age_differences}, deliberately select and investigate them in greater depth. However, it can be deduced from the research of these studies that these factors can influence the results of the studies and thus, in the worst case, lead to inconsistent results. 
Therefore, it can be argued that studies should always collect data on these factors and present the uncanny valley only in the context of these factors. Moreover, more research is needed in this direction to better define the effects of these influences on the uncanny valley to get a more accurate picture of the uncanny valley.

\newpage
\begin{sidewaystable}
%\centering
\setlength{\extrarowheight}{0pt}
\addtolength{\extrarowheight}{\aboverulesep}
\addtolength{\extrarowheight}{\belowrulesep}
\setlength{\aboverulesep}{0pt}
\setlength{\belowrulesep}{0pt}
\resizebox{\linewidth}{!}{%
\begin{tabular}{|l|l|l|l|l|l|l|l|l|l|l|l|} 
\toprule
\rowcolor[rgb]{0.753,0.753,0.753}  & \cite{uncanny_ambiguous} & \cite{uncanny_cliff} & \cite{uncanny_wall} & \cite{uncanny_infants} & \cite{uncanny_infant_discrimination} & \cite{childrens_responding} \\ 
\hline
{\cellcolor[rgb]{0.753,0.753,0.753}}Participants: & 56 & 58 & 100 & 96 & 42 & 77 \\ 
\hline
{\cellcolor[rgb]{0.753,0.753,0.753}}Female: & 13 & 28 & 8 & 46 & 22 & 38 \\ 
\hline
{\cellcolor[rgb]{0.753,0.753,0.753}}Male: & 43 & 30 & 92 & 50 & 20 & 39 \\ 
\hline
{\cellcolor[rgb]{0.753,0.753,0.753}}Age: & 17-35 years & 18-41 years & not disclosed & 6-12 months & 6-14 months & not disclosed \\ 
\hline
{\cellcolor[rgb]{0.753,0.753,0.753}}Culture: & Indonesian & \begin{tabular}[c]{@{}l@{}}presumed~Japanese\\(not directly disclosed)\end{tabular} & \begin{tabular}[c]{@{}l@{}}presumed American\\(not directly disclosed)\end{tabular} & mostly Caucasian & not disclosed & \begin{tabular}[c]{@{}l@{}}presumed American\\(not directly disclosed)\end{tabular} \\ 
\hline
{\cellcolor[rgb]{0.753,0.753,0.753}}Occupation: & \begin{tabular}[c]{@{}l@{}}university students,\\young professionals,\\government workers~\end{tabular} & \begin{tabular}[c]{@{}l@{}}participants were\\associated with a\\university in the\\Kyoto district of\\Japan\end{tabular} & \begin{tabular}[c]{@{}l@{}}university students\\from the School of\\Games Computing\\and Creative\\Technologies at\\Bolton University~ ~\end{tabular} &  &  & \\
\bottomrule
\end{tabular}
}
\caption{Various participation groups.}
\label{tab:participation_1}
%\end{sidewaystable}

%\begin{sidewaystable}
%\centering
\setlength{\extrarowheight}{0pt}
\addtolength{\extrarowheight}{\aboverulesep}
\addtolength{\extrarowheight}{\belowrulesep}
\setlength{\aboverulesep}{0pt}
\setlength{\belowrulesep}{0pt}
\resizebox{\linewidth}{!}{%
\begin{tabular}{|l|l|l|l|l|l|l|l|l|l|l|l|} 
\toprule
\rowcolor[rgb]{0.753,0.753,0.753} & \cite{prior_exposure_robots} & \cite{age_differences} & \cite{review_1} & \cite{review_2} & \cite{review_3} \\ 
\hline
{\cellcolor[rgb]{0.753,0.753,0.753}}Participants: & 86 & 255 & 30 & 72 & 20 \\ 
\hline
{\cellcolor[rgb]{0.753,0.753,0.753}}Female: & not disclosed & 157 & 9 & 46 & 10 \\ 
\hline
{\cellcolor[rgb]{0.753,0.753,0.753}}Male: & not disclosed & 98 & 21 & 26 & 10 \\ 
\hline
{\cellcolor[rgb]{0.753,0.753,0.753}}Age: & not disclosed & 18-65+ years & median age: 26 years & 18-49 years & not disclosed \\ 
\hline
{\cellcolor[rgb]{0.753,0.753,0.753}}Culture: & not disclosed & mostly Taiwanese & not disclosed & \begin{tabular}[c]{@{}l@{}}presumed American\\(not directly disclosed)\end{tabular} & not disclosed \\ 
\hline
{\cellcolor[rgb]{0.753,0.753,0.753}}Occupation: & not disclosed & not disclosed & not disclosed & \begin{tabular}[c]{@{}l@{}}recruited from Tufts\\University and the\\surrounding community\end{tabular} & \begin{tabular}[c]{@{}l@{}}recruited from\\a combination\\of university\\mailing lists\\and prolific\\participant\\recruitment\end{tabular} \\
\bottomrule
\end{tabular}
}
\caption{Various participation groups.}
\label{tab:participation_2}
\end{sidewaystable}