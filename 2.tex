\chapter{An Uncanny Cliff?}
In Masahiro Mori's original hypothesis, one can see that if the appearance and the movements of an entity become indistinguishable from humans it is possible for the entity to escape the uncanny valley. Moreover it is even possible that the affinity for an entity which has overcome the uncanny valley exceeds the affinity of entities which have not yet fallen into the uncanny valley. If this hypotheses holds true, it would have major implications for robotics and other scientific fields, as they must strive to design entities whose looks and movements are as similar as possible to that of a human being.\\
On the contrary it may be possible that the uncanny valley rather resembles an uncanny cliff in which the affinity for entities with perfect or near perfect human likeness is in general lower than that of entities that did not fall into the uncanny valley. In this hypothesis, it would not be advantageous to strive for perfect human likeness.\\
A study by Christoph Bartneck, Takayuki Kanda, Hiroshi Ishiguro and Norihiro Hagita, tried to plot the uncanny valley with particular emphasis on the last ascending section of the curve with more extensive measurements. Furthermore, the referred study also dealt with the question whether highly human-like androids are perceived more likeable when they are being framed as robots. \cite{4415111}

\section{Methode}
In the study a framing and a anthropomorphism experiment were conducted. Framing contained three conditions: human, robot and none. Anthropomorphism consisted of four conditions real human, manipulated human, computer graphic and android. Additionally only in the robot framing condition two additional anthropomorphisms were present: humanoid and pet robot. For the experiment only pictures of entities that either exist or which are extremely similar to existing entities were chosen. 
With a questionnaire the human likeness and the likeability of the stimuli was measured.
To ensure the framing conditions of this study three different questionnaires with different framing of the pictures where created. In each different framing of the questionnaire the pictures were either framed as human, robot or only as a face for a neutral comparison. For each anthropomorphism category three different pictures where shown to the participants. 
58 People participated in the study aged between 18 and 41 years. 28 of which were female and 30 were male.

%TODO Proceedings
%TODO Results

%TODO also maybe include uncanny wall in this chapter





