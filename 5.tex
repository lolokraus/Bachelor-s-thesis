\chapter{Exploring the Uncanny Valley in Humanlike Robot Recruiters}
Digatilisation is progressing rapidly in human resources. Many processes are already automated and the use of artificial intelligence is also becoming increasingly common. Especially in the recruitment process, a new approach called robot recruiting is being applied. Robot recruitment describes a semi-automatic process in which algorithms and artificial intelligence take over parts of the normal recruiting process for example the assessment and selection of applicants \cite{robot_recruiting}. The algorithms are trained with specific data sets, for each specific use case, to find answers and solve problems independently \cite{robot_recruiting_avant}. Sometimes also only simple rule-based systems are used \cite{robot_recruiting_avant}. In contrast to programs that are based on artificial intelligence and thus approach their tasks dynamically, where not every step has to be defined by a human being, rule-based system have to be specified precisely and each instruction is defined in advance \cite{robot_recruiting_avant}. Robot recruiting is hoped to shorten the time consuming process of personnel recruitment and to employ a more neutral way by evaluating the applicants through algorithms \cite{robot_recruiting}.\\
We are all familiar with websites that offer a chat window where you can talk to supposed employees. Often these 'employees' are in fact algorithms with witch you can communicate and they can perform a variety of simple tasks \cite{robot_recruiting}. These are among the best known applications that are already using robot recruiting. This is also how you can think of the recruitment processes using robot recruitment. An algorithm takes over the simpler communication and carries out a sorting and admission and possibly also an exclusion of applicants \cite{robot_recruiting}. However, with the rapid development of information technology, robot recruiters can take on even more complex tasks and they sometimes even take on a human form. A Russian start-up has developed a robot recruitment system called robot vera, with which companies can conduct telephone interviews with applicants \cite{robot_recruiting}. The robot talks to the applicants self-sufficiently and responds to their questions. During the telephone interviews the robot takes on a female human form \cite{robot_recruiting}. But this is where the uncanny valley comes into play.\\
With the help of a survey based on rating different human-like figures with a questionnaire, I would like to explore how the applicants' behaviour changes in the recruitment process if they feel an uncanny valley towards a human-looking robot recruiter.
Furthermore, I would like to find out whether such visualised robot recruiters are an advantage or a disadvantage during a robot recruitment process by means of the inclination towards or rejection of such visualised robot recruiters. Based on the results of my study, I would then like to make a recommendation on the human-like appearance that such visualised robot recruitment systems should aim for. 