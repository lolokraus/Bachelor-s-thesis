\chapter{Exploring the Uncanny Valley in Human-Like Robot Recruiters}
Digitalisation is progressing rapidly in human resources processes. Many procedures are already automated, and the use of artificial intelligence is also becoming increasingly common. Especially in the recruitment process, a new approach called robot recruiting is being applied. Robot recruitment describes a semi-automatic process in which algorithms and artificial intelligence take over parts of the typical recruiting process, such as assessing and selecting applicants \cite{robot_recruiting}. By analysing vast amounts of data, these algorithms are trained to assist companies in making predictions and decisions on future employees \cite{robot_recruiting_scholar}. This assistance is hoped to shorten the time-consuming process of personnel recruitment and employ a more neutral way of evaluating the applicants \cite{robot_recruiting_scholar}.\\
The majority of the people are familiar with websites that offer a chat window where one can talk to supposed employees. Often these 'employees' are, in fact, algorithms with which you can communicate, and they can perform a variety of simple tasks \cite{robot_recruiting}. These algorithms are among the best-known applications similar to the intention of robot recruiting. When applying this concept to the recruitment processes, an algorithm takes over the simpler communication and carries out sorting and admission and possibly excluding applicants \cite{robot_recruiting}. However, with the rapid development of information technology, robot recruiters can take on even more complex tasks, such as conducting a job interview \cite{robot_recruiting}. For this task, however, the algorithms need a medium with which to communicate with candidates, and here a human-like figure in the form of an online character or a robot is often chosen. One example of this is a Russian start-up which has developed a robot recruitment system called robot vera, with which companies can conduct telephone interviews with applicants \cite{robot_recruiting}. The robot talks to the applicants self-sufficiently and responds to their questions. During the telephone interview, the robot takes on a female human form \cite{robot_recruiting}.\newpage
With the development of a human-like appearance of robot recruiters, the effect of the uncanny valley on them also becomes relevant. If the appearance of a robot recruiter falls into the uncanny valley, it could affect the applicant's acceptance of the recruiter during the interview. This could negatively impact the applicant's behaviour and therefore also his performance during the application process.
With the help of a survey based on rating different human-like figures, this thesis explores how applicants' feelings towards different design of robot recruiters vary. The thesis also attempts to demonstrate an uncanny valley in very but not perfectly human-like robot recruiters. In addition, an attempt is made to make recommendations for the degree of human likeness of such entities based on the results of the survey.  

%visualised robot recruiters are an advantage or a disadvantage during a robot recruitment process by means of the inclination towards or rejection of such visualised robot recruiters.


\section{Design}%Methode
\section{Participants}
\section{Procedure}
\section{Measures}
\section{Analyses}
\section{Results}
\section{Discussion}
