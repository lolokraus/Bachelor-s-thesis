\chapter{The Original Uncanny Valley}

The uncanny valley was first hypothesized by Masahiro Mori, a robotics professor at the
Tokyo Institute of Technology. In an essay, he published in 1970, he theorised that as the appearance of a robot 
is approaching, but failing to attain, a lifelike appearance the empathy of a person towards the robot would 
abruptly shift to revulsion.\\
At first, little attention was paid to his work however as technology evolved and more and more human-like robots were being
built the effects of the uncanny valley have gained more significance in robotics and other scientific fields. 
This paper is going to mainly focus on the effects of the Uncanny Valley in relation to human-robot interaction.\cite{6213238}

\section{The Uncanny Valley with regards to the appearance}
To explain the uncanny valley in more detail Masahiro Mori choose in his work three different robots with varying human 
likeness in regards to their appearance and described our feeling towards these robots. First he looked at a industrial robot
which design is solely base on functionality. Given their non-existent resemblance to humans and their exclusively funtional
existence people feel hardly any affinity for them.\\
